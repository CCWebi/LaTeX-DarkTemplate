% -------------
% Módulo con los paquetes default de la plantilla.
% -------------

\usepackage{luacode} % Ejecutar lua-code
\usepackage{silence}
\usepackage[
backend=biber,
style=numeric,
sortlocale=author,
]{biblatex} % Bibliografías y referencias
\usepackage{csquotes} % Comillas para biblatex
\bibliography{sample.bib} % Archivo con las referencias
\usepackage{tikz} % Para gráficar elementos
\usetikzlibrary {automata,positioning,arrows.meta} % Necesarios para árboles, automatas, etc

\usepackage[dvipsnames, table]{xcolor} % Colores (colores dvisps y para NiceMatrix)
\usepackage{luacolor} % LuaLaTeX colors optim
\usepackage{lua-ul} % LuaLaTeX highlighting and strikeout
\usepackage{soul} % pdflatex highlighting

\usepackage{calc} % Cálculos JIT
\usepackage{bold-extra} % Small bold/TT caps
\usepackage{ragged2e} % Justificación del texto

\usepackage{amsmath,amssymb,amsfonts,mathrsfs,commath} % Símbolos y fuenets matemáticas
\usepackage{siunitx} % Unidades del SisInt (útil con floats)
\usepackage{systeme} % Sistemas de ecuaciones

\usepackage{multicol} % Multicolumnas en tablas
\usepackage{multirow} % Multirenglones en tablas
\usepackage{colortbl} % Color en tablas default
\usepackage{nicematrix} % Para tablas GOD 
\usepackage{longtable} % Tablas multipágina
\usepackage{booktabs} % Tablas simples pero más bonitas (sin verticales)
\usepackage{worldflags} % Banderas 4K

\usepackage{graphicx} % Gráficas
\usepackage{lipsum} % Lorem Ipsum
\usepackage{fancyhdr} % Headers y footers
\usepackage{listings} % Código

\usepackage{comment} % Environment comentario
\usepackage{wrapfig} % Texto pegado a imagenes
\usepackage{adjustbox} % Ajustar tablas a tamaño \begin{adjustbox}{width=\textwidth}
\usepackage{parskip} % Evita usar // y separa automáticamente
\usepackage{enumitem} % Format enumerates/itemize
\usepackage{caption} % Format captions
\usepackage{subcaption} % Captions distintos en misma figura
\usepackage{hyperref} % Referenicias (in document)
\usepackage{float} % Float (opción H "h pero obligado")
\usepackage{etoolbox} % ifs
\usepackage{setspace} % Interlineado
\usepackage{fancyvrb} % Formatear verbatim

